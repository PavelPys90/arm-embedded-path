% Showcase Präsentation mit SimSchoolDark Theme
\documentclass{beamer}

% Theme laden
\usetheme{SimSchoolDark}

% Für Python Code Highlighting
\usepackage{listings}
\usepackage{xcolor}
\usepackage{tabu}
\usepackage{booktabs}
\usepackage{tabularx}
\usepackage[T1]{fontenc}

\definecolor{codegreen}{rgb}{0,0.6,0}
\definecolor{codegray}{rgb}{0.45,0.45,0.45}
\definecolor{codepurple}{rgb}{0.58,0,0.82}
\definecolor{backcolour}{rgb}{0.96,0.96,0.96}

\lstdefinestyle{mystyle}{
	backgroundcolor=\color{backcolour},
	commentstyle=\color{codegreen}\itshape,
	keywordstyle=\color{magenta}\bfseries,
	stringstyle=\color{codepurple},
	numberstyle=\tiny\color{codegray},
	basicstyle=\ttfamily\small,  % kompakter als footnotesize, aber noch gut lesbar
	breakatwhitespace=false,
	breaklines=true,
	captionpos=b,
	keepspaces=true,
	numbers=none,                % ich empfehle in Beamer lieber keine Zeilennummern
	showspaces=false,
	showstringspaces=false,
	showtabs=false,
	tabsize=4,                   % einheitlich 4 Spaces für Tabs
	frame=single,                % optional: Rahmen um den Code
	rulecolor=\color{gray!50},   % dezente Rahmenfarbe
	aboveskip=5pt,
	belowskip=5pt,
	xleftmargin=4pt, xrightmargin=4pt,
	columns=fullflexible         % sorgt für gleichmäßige Zeichenabstände
}

\lstset{style=mystyle}

% Python Code Style
%\lstset{
%	language=Python,
%	basicstyle=\ttfamily\small,
%	keywordstyle=\color{simschooldarkest}\bfseries,
%	stringstyle=\color{simschoolmedium},
%	commentstyle=\color{simschoollight}\itshape,
%	numberstyle=\tiny\color{simschoollight},
%	numbers=left,
%	breaklines=true,
%	frame=single,
%	rulecolor=\color{simschoolmedium},
%	backgroundcolor=\color{simschoolbg!50},
%	showstringspaces=false,
%	tabsize=4
%}

% Präsentationsinfos
\title{ARM-Embedded-Path}
\subtitle{Hardwaretests for STMF103CxT6}
\author{Pavel Pys}
\date{\today}
%\institute{}
\begin{document}
\begin{frame}{}
	\maketitle
\end{frame}
\begin{frame}{Overview}
	\tableofcontents
\end{frame}
\section{Motivation}
% Slide 1: Title Slide
\begin{frame}
	\begin{center}
		\Huge
		\textbf{Why Chip Verification Matters}
		
		\vspace{1cm}
		
		\Large
		Identifying Counterfeit STM32F103 Microcontrollers
		
		\vspace{1.5cm}
		
		\normalsize
		Focus: STM32F103C6T6A \& STM32F103C8T6
	\end{center}
\end{frame}

% Slide 2: The Counterfeit Problem
\begin{frame}{The Reality}{Counterfeit Chips in the Supply Chain}
	\textbf{The Problem is Real and Growing:}
	\begin{itemize}
		\item ERAI reported \textbf{786} cases in \textbf{2023} and \textbf{1,055} in \textbf{2024} (highest level since 2015)%
		\footnotesize~[ERAI 2023/2024]%
		\item "Blue Pill" boards with STM32F103 are particularly affected by clones/drop-in replacements
		\item Occurrences also on major marketplaces (AliExpress, eBay, Amazon)
		\item Even reputable distributors can unintentionally supply non-original parts
	\end{itemize}
	
	\medskip
	\textbf{Common Scenarios:}
	\begin{itemize}
		\item Remarking: \textit{C6} sold as \textit{C8} (lower density → higher margin)
		\item Wafer rejects, refurbished parts
		\item Pin-compatible clones with different behavior (GD32, APM32, CH32, CS32)
		\item Completely different chips with counterfeit markings
	\end{itemize}
	
	\medskip
	\textcolor{red}{\textbf{Bottom Line:}} Never trust markings alone — always verify!
\end{frame}
% Slide 3: Why This Matters for Development
\begin{frame}{Impact on Development}{Why Should You Care?}
	\textbf{1. Debugging Nightmares:}
	\begin{itemize}
		\item Intermittent bugs that make no sense
		\item Code works on one board, fails on another
		\item Hours wasted chasing phantom issues
		\item Features that simply don't work (DMA, USB, ADC channels)
	\end{itemize}
	
	\medskip
	\textbf{2. Product Reliability:}
	\begin{itemize}
		\item Field failures that are impossible to reproduce
		\item Temperature/voltage instability
		\item Unexpected resets or crashes
		\item Warranty claims and customer dissatisfaction
	\end{itemize}
	
	\medskip
	\textbf{3. Time and Money:}
	\begin{itemize}
		\item Wasted development hours debugging non-genuine chips
		\item Production delays when issues discovered late
		\item Cost of rework and board replacements
	\end{itemize}
\end{frame}

% Slide 4: Example Scenario - Flash Size Mismatch
\begin{frame}[fragile]{Example Scenario}{Flash Size Verification}
	\textbf{Common Issue:} Tooling reports 64/128\,KB, actual usable size differs
	
	\smallskip
	\textbf{Verify via Flash Size Register (authoritative):}
	\textbf{Real-World Blue Pill Observations (Community):}
	\begin{itemize}
		\item Tools/programmers sometimes report 128\,KB, Flash Size Register shows 64\,KB
		\item Clones (e.g., CS32) with different IDCODEs/USB behavior reported
		\item Writing beyond actual size causes corruption/crashes
	\end{itemize}
\end{frame}
% Slide 5: C6 vs C8 - What's the Difference?
\begin{frame}{STM32F103C6 vs C8}{Understanding the Variants}
	\begin{center}
		\small
		\begin{tabular}{|l|c|c|}
			\hline
			\textbf{Feature} & \textbf{C6T6A} & \textbf{C8T6} \\ \hline
			Flash Memory & 32 KB & 64 KB \\ \hline
			SRAM & 10 KB & 20 KB \\ \hline
			\textbf{DBGMCU DEV\_ID} & \textbf{0x412} (Low density) & \textbf{0x410} (Medium density) \\ \hline
			Density & Low & Medium \\ \hline
			Price (typical) & Lower & Higher \\ \hline
		\end{tabular}
	\end{center}

	\medskip
	\textbf{Why C6→C8 Remarking?}
	\begin{itemize}
		\item Price difference, same 48-pin LQFP package, identical marking area
	\end{itemize}
	
	\medskip
	\textbf{Common Drop-in Alternatives:}
	\begin{itemize}
		\item \textbf{GD32F103} (GigaDevice), \textbf{APM32F103} (Geehy), \textbf{CH32F103} (WCH), \textbf{CS32F103}
		\item Pin-compatible but \emph{not} bit-identical; timing/peripherals may differ
	\end{itemize}
\end{frame}

\begin{frame}{Failure Modes}{Common Issues with Non-Genuine/Clone Chips}
	\textbf{Missing/Deviating Features (Observations):}
	\begin{itemize}
		\item DMA channels unreliable
		\item Fewer/different number of ADC channels
		\item USB enumeration/function differs (community reports, e.g., CS32)
		\item CAN/RTC oscillator issues
	\end{itemize}
	
	\medskip
	\textbf{Electrical Characteristics:}
	\begin{itemize}
		\item Higher power consumption, different limits
		\item Temperature/power supply sensitivity
		\item Poorer EMC/EMI performance
	\end{itemize}
	
	\medskip
	\textbf{Timing/Performance:}
	\begin{itemize}
		\item HSI accuracy/clock drift
		\item Flash wait states required at high clock frequencies
		\item Interrupt latencies/peripheral timing differ
	\end{itemize}
\end{frame}

% Slide 7: The Business Case
\begin{frame}{Business Impact}{Example Cost Calculation}
	\textbf{Example Scenario (Assumptions):}
	\begin{itemize}
		\item Small production batch: 100 units
		\item Non-genuine chips cause 20\% field failure rate
		\item Cost per RMA: \$50 (shipping, diagnosis, replacement)
		\item Engineering debugging time: 40 hours @ \$75/hr
	\end{itemize}
	
	\medskip
	\textbf{Example Cost Breakdown:}
	\begin{itemize}
		\item 20 RMA returns @ \$50 = \$1,000
		\item Engineering time = \$3,000
		\item \textbf{Total example loss: \$4,000}
		\item Chip cost "savings": ~\$100
	\end{itemize}
	
	\medskip
	\textbf{Additional Hidden Costs:}
	\begin{itemize}
		\item Lost customer confidence
		\item Damage to brand reputation
		\item Additional QA processes needed
		\item Production delays
	\end{itemize}
	
	\medskip
	\textit{Note: This is an illustrative example, actual costs vary by project}
\end{frame}

\begin{frame}{Legal \& Safety}{Beyond Just Technical Issues}
	\textbf{Regulatory Compliance (Examples):}
	\begin{itemize}
		\item \textbf{Aerospace:} \textbf{AS9100D} (8.1.4) requires counterfeit prevention processes; \textbf{SAE AS5553} specifies measures
		\item \textbf{Automotive:} \textbf{ISO 26262} requires comprehensive \emph{traceability} and controlled processes 
		(\(\Rightarrow\) non-traceable parts endanger the safety case)
		\item \textbf{Medical:} EU-\textbf{MDR}/IVDR with UDI traceability requirements
		\item \textbf{Industrial:} CE/UL require documented supply chain
	\end{itemize}
	
	\medskip
	\textbf{Safety-Critical Applications:} Medical devices, automotive, industrial safety, aviation
	
	\medskip
	\textbf{Legal Exposure:}
	\begin{itemize}
		\item Product liability for damages caused by non-compliant parts
		\item Contract/compliance violations (traceability requirements)
	\end{itemize}
\end{frame}

\begin{frame}{Our Approach}{Multi-Layer Verification Strategy}
	\textbf{Why Multiple Tests?} Single tests can be misleading; layering increases security
	
	\smallskip
	\textbf{1. DBGMCU Device ID (0xE0042000):}
	\begin{itemize}
		\item DEV\_ID \textbf{0x410} (Medium), \textbf{0x412} (Low) - quick plausibility check
		\item Clones may provide same DEV\_ID; \textbf{REV\_ID} often differs (e.g., GD32)
	\end{itemize}
	
	\smallskip
	\textbf{2. Flash Size Register (0x1FFFF7E0):}
	\begin{itemize}
		\item Authoritative size in kB, more forgery-resistant than tool heuristics
	\end{itemize}
	
	\smallskip
	\textbf{3. Unique ID (0x1FFFF7E8):}
	\begin{itemize}
		\item 96-bit UID for serial number/batch analysis
	\end{itemize}
	
	\smallskip
	\textbf{4. Peripheral Testing:}
	\begin{itemize}
		\item Verify specific peripheral behavior (USB, ADC timing, etc.)
		\item Compare against known genuine chip characteristics
	\end{itemize}
\end{frame}
\begin{frame}{Functional Verification}{Testing Actual Hardware}
	\textbf{4. DMA Functional Test:}
	\begin{itemize}
		\item Verifies actual peripheral functionality (Mem\,$\leftrightarrow$\,Periph)
		\item Catches missing/deviating DMA implementations
	\end{itemize}
	
	\medskip
	\textbf{5. Additional Quick Tests:}
	\begin{itemize}
		\item ADC channels and accuracy
		\item Timer/interrupt timing
		\item USB enumeration (if used)
		\item Clock accuracy (HSI/HSE drift)
	\end{itemize}
	
	\medskip
	\textbf{Best Practice:}
	\begin{itemize}
		\item Combine ID checks \emph{plus} functional tests
		\item Document/archive results (traceability)
		\item Use conservative timings (derivatives may have different default clocks)
	\end{itemize}
\end{frame}

\begin{frame}{Today's Workshop}{Practical Chip Verification}
	\textbf{Hands-On Exercises:}
	\begin{enumerate}
		\item Read \texttt{DBGMCU->IDCODE} (\textbf{0xE0042000}) — DEV\_ID/REV\_ID
		\item Verify flash size via \texttt{0x1FFFF7E0}
		\item Read UID from \texttt{0x1FFFF7E8}
		\item DMA functionality test
	\end{enumerate}
	
	\medskip
	\textbf{Deliverable:} Test firmware, result documentation, decision tree (OK/Reject)
\end{frame}

\begin{frame}{Prevention Strategy}{How to Avoid Counterfeits}
	\textbf{Procurement (Best Practice):}
	\begin{itemize}
		\item Prefer \textbf{authorized} distributors: Digi-Key, Mouser, Farnell/element14, ST eStore
		\item Verify authorization on ST website; request CoC/traceability documentation
		\item Be cautious with prices significantly below market rate
	\end{itemize}
	
	\medskip
	\textbf{Red Flags:}
	\begin{itemize}
		\item No manufacturer packaging/trays
		\item Lack of traceability documentation
		\item Unusual fonts/markings; "pulls/refurbs" for new designs
	\end{itemize}
	
	\medskip
	\textbf{Nevertheless:} Establish QA verification before production \& as part of bring-up process
\end{frame}
\section{Configuration}
% ========================
% Konfiguration — Blue Pill (STM32F103C6T6A)
% ========================

% Slide 1 — Ziel & Setup
\begin{frame}{Configuration}{Goal \& Setup}
	\textbf{Preparation Goal:}
	\begin{itemize}
		\item ST-LINK + STM32CubeProgrammer: Check \textit{Connect} and \textbf{Device ID}
		\item UART output of test firmware (DEV\_ID/FlashSize/UID)
		\item Small \textbf{DMA} test (USART1 \textrightarrow{} DMA1)
	\end{itemize}
	
	\medskip
	\textbf{Required:}
	\begin{itemize}
		\item Blue Pill (STM32F103\textbf{C6}T6A), ST-LINK/V2(/V3), 3.3V UART-TTL adapter
		\item USB cable, 4 Dupont wires for SWD, 3 wires for UART
	\end{itemize}
	
	\medskip
	\textbf{Premise:} \textit{Configuration} now; \textit{implementation/code} in next section.
\end{frame}
% Slide 2 — SWD Wiring (ST-LINK)
\begin{frame}{Configuration}{SWD — ST-LINK Wiring}
	\textbf{Pins (Blue Pill / STM32F103):}
	\begin{itemize}
		\item \textbf{SWDIO} \textrightarrow{} \textbf{PA13}, \textbf{SWCLK} \textrightarrow{} \textbf{PA14}
		\item \textbf{GND} \textrightarrow{} GND, \textbf{3V3} \textrightarrow{} 3V3 (reference)
	\end{itemize}
	
	\medskip
	\textbf{Notes:}
	\begin{itemize}
		\item Set interface to \textbf{SWD} in tool (not JTAG)
		\item Common ground
		\item Power board with 3.3V (from ST-LINK or external)
	\end{itemize}
\end{frame}

% Slide 3 — UART-TTL Wiring (USART1)
\begin{frame}{Configuration}{UART — 3.3V TTL (USART1)}
	\textbf{Pins:}
	\begin{itemize}
		\item \textbf{PA9 = USART1\_TX} \textrightarrow{} to \textbf{RX} of TTL adapter
		\item \textbf{PA10 = USART1\_RX} \textrightarrow{} to \textbf{TX} of TTL adapter
		\item \textbf{GND \textrightarrow{} GND}; do not use 5V TTL
	\end{itemize}
	
	\medskip
	\textbf{Terminal recommendation:} 115200 8N1, enable CR/LF
\end{frame}

% Slide 4 — CubeProgrammer: Connect & Device Info
\begin{frame}{Configuration}{STM32CubeProgrammer — Connect}
	\textbf{Procedure:}
	\begin{enumerate}
		\item \textit{Interface} = \textbf{ST-LINK}, \textit{Frequency} moderate (e.g., 400-1800 kHz)
		\item Click \textbf{Connect} \textrightarrow{} Device info: \textit{Device ID}, \textit{Revision}, \textit{Flash size}
		\item Optional: View \textit{OB} (Option Bytes) only, do \textit{not} modify
	\end{enumerate}
	
	\medskip
	\textbf{Goal of this phase:} Visual verification that ST-LINK and chip communicate correctly (DEV\_ID plausible).
\end{frame}

% Slide 5 — Which IDs/Signatures are relevant?
\begin{frame}{Configuration}{Electronic Signature — Addresses}
	\textbf{In Firmware (for next section):}
	\begin{itemize}
		\item \textbf{DBGMCU\_IDCODE} @ \texttt{0xE0042000} \textrightarrow{} \textit{DEV\_ID}/\textit{REV\_ID}
		\item \textbf{Flash Size} @ \texttt{0x1FFFF7E0} \textrightarrow{} Size in KByte (e.g., 32/64)
		\item \textbf{UID (96-bit)} starting @ \texttt{0x1FFFF7E8} \textrightarrow{} Serial/Batch analysis
	\end{itemize}
	
	\medskip
	\textbf{Practice:} DEV\_ID is a quick plausibility check; \textbf{Flash Size} is authoritative.
\end{frame}

% Slide 6 — Clock/Reset Assumptions for Tests
\begin{frame}{Configuration}{Clocks \& Reset — Assumptions}
	\textbf{Sufficient for configuration:}
	\begin{itemize}
		\item \textbf{HSI 8MHz} (reset default), no PLL
		\item \textbf{RCC}: activate only required peripheral clocks:
		\begin{itemize}
			\item \textbf{APB2}: GPIOA, AFIO, USART1
			\item \textbf{AHB}: DMA1 (for DMA test)
		\end{itemize}
		\item \textbf{Boot0}=0, \textbf{Boot1}=0 (Flash boot)
	\end{itemize}
\end{frame}

% Slide 7 — USART1: Prerequisites for UART Output
\begin{frame}{Configuration}{USART1 — Configuration Prerequisites}
	\textbf{RCC:} Enable \textbf{GPIOA}, \textbf{AFIO}, \textbf{USART1}
	
	\medskip
	\textbf{Pins:}
	\begin{itemize}
		\item \textbf{PA9} as \textbf{AF Push-Pull} (TX), \textbf{PA10} as \textbf{Input Floating} (RX)
		\item Optional Remap: \textbf{USART1\_REMAP}=0 (Default: TX/PA9, RX/PA10)
	\end{itemize}
	
	\medskip
	\textbf{USART1 Basic Parameters:}
	\begin{itemize}
		\item 115200 baud @ 8MHz HSI (baud rate calculated later), \textbf{8N1}
		\item Set \textbf{UE/TE/RE} (\textit{USART\_CR1})
	\end{itemize}
\end{frame}

% Slide 8 — DMA: Goals & Mapping (USART1)
\begin{frame}{Configuration}{DMA — Goals \& Mapping (USART1)}
	\textbf{Goal:} CPU-offloaded transfer, e.g., send string via DMA to USART1
	
	\medskip
	\textbf{RCC:} Enable \textbf{DMA1}
	
	\medskip
	\textbf{Channel Assignment (F1):}
	\begin{itemize}
		\item \textbf{USART1\_TX} \textrightarrow{} \textbf{DMA1 Channel 4}
		\item \textbf{USART1\_RX} \textrightarrow{} \textbf{DMA1 Channel 5}
	\end{itemize}
	
	\medskip
	\textbf{USART1:} Set \textbf{CR3.DMAT/DMAR} (TX/RX via DMA)
\end{frame}

% Slide 9 — DMA: Configuration Checklist (without Code)
\begin{frame}{Configuration}{DMA — Configuration Checklist}
	\textbf{TX (DMA1 Ch4):}
	\begin{itemize}
		\item \textbf{DIR} = Memory\textrightarrow{}Peripheral
		\item \textbf{PADDR} = \&USART1\textrightarrow{}DR, \textbf{MADDR} = data buffer
		\item \textbf{MINC}=1, \textbf{PSIZE/MSIZE}=8 bit, \textbf{TCIE} optional
	\end{itemize}
	
	\medskip
	\textbf{RX (DMA1 Ch5):}
	\begin{itemize}
		\item \textbf{DIR} = Peripheral\textrightarrow{}Memory
		\item \textbf{CIRC} optional for continuous reception
	\end{itemize}
	
	\medskip
	\textbf{Start:} Configure channel \textrightarrow{} set \textbf{EN}; monitor \textbf{USART1.TXE/TC} flags.
\end{frame}

% Slide 10 — Typical Pitfalls
\begin{frame}{Configuration}{Typical Pitfalls}
	\begin{itemize}
		\item \textbf{Levels}: UART adapter must be \textbf{3.3V TTL} (not 5V)
		\item \textbf{SWD}: Wrong interface or no 3.3V reference \textrightarrow{} no connect
		\item \textbf{AFIO Remap}: USART1\_REMAP accidentally set \textrightarrow{} pins no longer match
		\item \textbf{DMA}: Wrong channel (not Ch4/Ch5) or writing \textit{value} instead of address to \textbf{PADDR}
		\item \textbf{Clock}: Forgot RCC enable for GPIO/USART/DMA
	\end{itemize}
	
	\medskip
	\textbf{Ready for "Implementation":} Code snippets next (DEV\_ID/FlashSize/UID \textrightarrow{} UART, then DMA-TX).
\end{frame}
\section{Manual}
% =======================
% Blue Pill (STM32F103C6T6A): Setup, Konfiguration & Interpretation
% =======================

% Slide A: Hardware & Tools
\begin{frame}{A. Hardware \& Tools}{What you need \& how to connect}
	\textbf{Boards/Tools}
	\begin{itemize}
		\item Blue Pill \textbf{STM32F103C6T6A} (Low-Density, 32 KB Flash)
		\item ST-LINK/V2 (or V3) for SWD (Debug/Flash)
		\item USB-UART (3.3V TTL) for console output
	\end{itemize}
	
	\medskip
	\textbf{Wiring (Minimum)}
	\begin{itemize}
		\item \textit{SWD}: SWDIO\,$\rightarrow$\,PA13, SWCLK\,$\rightarrow$\,PA14, 3V3, GND
		\item \textit{USART1}: PA9\,(TX)$\rightarrow$\,RX(USB-TTL), PA10\,(RX)$\leftarrow$\,TX(USB-TTL), GND
	\end{itemize}
	
	\medskip
	\textbf{Software}
	\begin{itemize}
		\item STM32CubeIDE (includes CubeMX)
		\item STM32CubeProgrammer (ST-LINK drivers/flasher)
		\item (Optional) IDE Serial Terminal Plugin ("TM Terminal")
	\end{itemize}
\end{frame}

% Slide B: New Project
\begin{frame}{B. New Project in STM32CubeIDE}{Project Setup (CubeMX integrated)}
	\textbf{Setup}
	\begin{enumerate}
		\item \textit{File $\rightarrow$ New $\rightarrow$ STM32 Project} $\Rightarrow$ MCU: \texttt{STM32F103C6Tx}
		\item Assign project name, Toolchain = STM32CubeIDE
	\end{enumerate}
	
	\medskip
	\textbf{Why CubeIDE?}
	\begin{itemize}
		\item Peripheral configuration \& code generator (\texttt{.ioc})
		\item Integrated debugging via ST-LINK
		\item Easy to add Eclipse plugins (Terminal)
	\end{itemize}
\end{frame}

% Slide C: CubeMX Configuration
\begin{frame}{C. CubeMX Configuration (.ioc)}{Minimum for ID/UID/USART-DMA Demo}
	\textbf{System}
	\begin{itemize}
		\item \textit{SYS}: \textbf{Debug = Serial Wire} (SWD enabled)
		\item \textit{Clock}: HSI 8 MHz (sufficient for start), later optionally HSE$\rightarrow$PLL (72 MHz)
	\end{itemize}
	
	\medskip
	\textbf{USART1 (PA9/PA10)}
	\begin{itemize}
		\item Mode: Asynchronous, 115200 8N1, Oversampling=16, no HW flow control
		\item \textit{DMA Settings}: USART1\_TX $\rightarrow$ DMA1 Ch4, USART1\_RX $\rightarrow$ DMA1 Ch5
	\end{itemize}
	
	\medskip
	\textbf{Registers we will read}
	\begin{itemize}
		\item \texttt{DBGMCU\_IDCODE @ 0xE0042000} (DEV\_ID, REV\_ID)
		\item \texttt{FLASH\_SIZE @ 0x1FFFF7E0} (Size in kB)
		\item \texttt{UID[95:0] @ 0x1FFFF7E8} (3$\times$32 Bit)
	\end{itemize}
\end{frame}

% Slide D: Flashing & Debugging
\begin{frame}{D. Flashing \& Debugging with ST-LINK}{From Build to Running Firmware}
	\textbf{Connection}
	\begin{enumerate}
		\item ST-LINK to SWDIO/PA13, SWCLK/PA14, 3V3, GND
		\item In CubeIDE: \textit{Debug As $\rightarrow$ STM32 MCU Application}
	\end{enumerate}
	
	\medskip
	\textbf{CubeProgrammer (optionally visible)}
	\begin{itemize}
		\item Device Connect $\Rightarrow$ \textit{Device Information}: DEV\_ID, Flash Size
		\item ST-LINK drivers installed? (Windows: STSW-LINK009)
	\end{itemize}
	
	\medskip
	\textbf{Typical Pitfalls}
	\begin{itemize}
		\item BOOT0 must be 0 (Flash start), common ground (GND) not forgotten
		\item Reduce SWD frequency with long cables
	\end{itemize}
\end{frame}

% Slide E: Display UART Output
\begin{frame}{E. Display UART Output}{Console Setup}
	\textbf{In CubeIDE (Serial Terminal)}
	\begin{enumerate}
		\item (If not present) \textit{Help $\rightarrow$ Install New Software} $\Rightarrow$ "TM Terminal"
		\item \textit{Window $\rightarrow$ Show View $\rightarrow$ Terminal} $\Rightarrow$ \textit{Serial}
		\item Correct COM port, 115200, 8N1, no flow control
	\end{enumerate}
	
	\medskip
	\textbf{Alternatives}
	\begin{itemize}
		\item \texttt{screen}, PuTTY, \texttt{minicom}, etc.
	\end{itemize}
\end{frame}

% Slide F: Interpretation
\begin{frame}{Interpretation}{What do DEV\_ID, FlashSize, UID, BRR mean?}
	\textbf{Example (your logs)}
	\begin{itemize}
		\item \texttt{DBGMCU\_IDCODE = 0x10006412} $\Rightarrow$ \textbf{DEV\_ID 0x412} (Low-Density F10x), \textbf{REV\_ID 0x1000}
		\item \texttt{Flash size = 32 KB} $\Rightarrow$ matches \textbf{STM32F103C6}
		\item \texttt{UID} (96-bit) readable $\Rightarrow$ plausible serial identifier
		\item \texttt{PCLK2 = 8 MHz, BRR = 0x0045, Over = 16} $\Rightarrow$ \textbf{115200 baud} correct
	\end{itemize}
	
	\medskip
	\textbf{Classification}
	\begin{itemize}
		\item \textbf{0x412} is the Low-Density device ID (F101/F102/F103 with 16-32 KB)
		\item \textbf{Flash Size Register} provides actual size in kB
		\item \textbf{UID} fixed at 0x1FFFF7E8 (3$\times$32 Bit)
		\item \textbf{BRR} (with 16x oversampling): $\text{Baud}=\frac{f_{\text{PCLK}}}{16\cdot(\text{Mant}+\text{Frac}/16)}$
	\end{itemize}
\end{frame}
\end{document}